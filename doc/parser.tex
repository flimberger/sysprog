\chapter{Parser}
\label{chap:parser}

\section{Parsebaum}
\label{sec:parsetree}

\subsection{Aufbau}
Ein Parsebaum besteht aus Knoten.
Jeder Knoten hat ein rechtes und ein linkes Kind sowie einen Typ und eine Datensektion.

\begin{lstlisting}
typedef struct Node Node;
struct Node {
	Node *left;
	Node *right;
	Nodetype type;
	union {
		vlong  val;
		Symbol *sym;
		Optype op;
	} data;
};
\end{lstlisting}

\subsection{Knotentypen}

\subsubsection{Wurzelknoten}
\texttt{left} enthält die Liste aller Deklarationen,
\texttt{right} die Liste aller Ausdrücke.
Der Wurzelknoten wird durch die Konstante \texttt{NODE_ROOT} gekennzeichnet.

\subsubsection{Listenknoten}
Listenknoten werden sowohl für Deklarationen als auch Ausdrücke verwendet,
da sie sich lediglich in den Typen ihrer Blätter unterscheiden.
Jeder Listenknoten enthält in \texttt{left} ein Blatt
und in \texttt{right} entweder ein Blatt oder einen weiteren Listenknoten.
Listenknoten mit nur einem Element werden durch dieses Element ersetzt.
Ein Listenknoten wird durch \texttt{NODE_LIST} identifiziert.

\subsubsection{Deklarationsknoten}
Ein Deklarationsknoten enthält in \texttt{data.sym} einen Zeiger auf ein Element der Symboltabelle.
Wenn es sich bei der deklarierten Variable um ein Array handelt,
dann erzeugt \texttt{parsearray} einen Knoten vom Typ \texttt{NODE_ARRAY},
welcher in \texttt{left} eingehängt wird.
Ein Deklarationsknoten hat den Typ \texttt{NODE_DECL}.

\subsubsection{Arrayknoten}
Ein Arrayknoten enthält lediglich die Deklarationsgröße in \texttt{data.val}.
Ein Arrayknoten wird durch \texttt{NODE_ARRAY} gekennzeichnet.

\subsubsection{Zusweisungsknoten}
Ein Zusweisungsknoten speichert in \texttt{data.sym} einen Zeiger auf ein Symbol in der Symboltabelle,
sowie einen 
% TODO NODE_EXP
in \texttt{right}.
Wenn es sich um ein Array handelt,
dann wird in \texttt{left} ein weiterer
% TODO NODE_EXP
eingehängt,
welcher den Index des Arrays beschreibt.
Zuweisungsknoten sind vom Typ \texttt{NODE_ASSGN}.
% EXP: entweder identifier, constant, operator oder group

\subsubsection{Variablenknoten}
\texttt{NODE_IDENT}

\subsubsection{Konstantenknoten}
\texttt{NODE_CONST}

\subsubsection{Operatorknoten}
\texttt{NODE_OP}

\subsubsection{Ausgabeknoten}
Ein Ausgabeknoten enthält den auszugebenden Ausdruck in \texttt{left}
und wird durch den Typ \texttt{NODE_PRINT} gekennzeichnet.

\subsubsection{Eingabeknoten}
Ein Eingabeknoten enthält in \texttt{left} einen Variablenknoten,
in welchem die Eingabe gespeichert wird.
Er wird über \texttt{NODE_PRINT} identifiziert.

\subsubsection{Verzweigungsknoten}
Ein Verzweigungsknoten enthält in \texttt{left} einen
%TODO NODE_EXP
und in \texttt{right} einen Knoten ohne Typ (auch \texttt{NODE_NONE}),
welcher wiederrum die Ausdrücke für den \emph{if}- in \texttt{left}
und \emph{else}-Zweig in \texttt{right} enthält.
Dieser Knoten hat den Typ \texttt{NODE_IF}.

\subsubsection{Schleifenknoten}
Ein Schleifenknoten speichert die Schleifenbedingung in \texttt{left}
und den Rumpfausdruck in \texttt{right}.
Er wird durch \texttt{NODE_WHILE} gekennzeichnet.
