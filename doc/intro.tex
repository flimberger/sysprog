\chapter{Einführung}
\label{chap:intro}

\section{Einordnung}
\label{sec:category}
Die Aufgabe in der Laborveranstaltung ``Systemnahes Programmieren'' der
Hochschule Karlsruhe ist es,
einen einfachen Compiler zu erstellen.

Im folgenden wird die Lösung des ersten Aufgabenteils,
dem Erstellen des Scanners,
dokumentiert.

Die API-Dokumentation der einzelnen Komponenten liegt in Form von Manual-Pages
im Verzeichnis \texttt{doc/man/} innerhalb des Projektverzeichnisses vor.
Wie unter den meisten unixoiden Betriebssystemen üblich,
sind diese in verschiedene Sektionen geteilt,
wobei Programme in Sektion 1 und Bibliotheken in Sektion 2 angesiedelt sind.

\section{Übersicht}
\label{sec:overview}
Das erste Kapitel, die Einführung,
beinhaltet die thematische Einordnung,
diese Übersicht sowie eine Beschreibung der Aufgabe.
Das zweite Kapitel befasst sich mit der Implementierung der
Ein-/Ausgabebibliothek,
welche die Pufferung der Daten bereitstellt.
Das nächste Kapitel widmet sich kurz der kleinen Fehlerbibliothek,
welche in allen Programmen verwendet wird.
In den darauffolgenden Kapiteln werden die Symboltabelle und die von dieser
verwendeten Stringtabelle vorgestellt,
womit die Grundlagen für das letzte Kapitel gelegt werden,
welches den Scanner beschreibt.
Dieser Teil baut auf die vorrausgehenden Kapitel auf,
da er alle zuvor beschriebenen Komponenten verwendet.

\section{Aufgabenstellung}
\label{sec:description}
Die Zielplattform für des zu erstellenden Compilers ist Linux.
Die zu verwendende Sprache ist C++,
zum Buildmanagement wird GNU Make verwendet.
Diese Vorgaben wurden in dieser Implementierung insofern abgewandelt,
dass sowohl Code als auch Makefiles weitestgehend portabel gehalten sind
und anstatt C++ reines C verwendet wird.
Volle Portabilität ist jedoch nicht möglich,
da in der Aufgabenstellung die Verwendung des \texttt{O\_DIRECT}-Flags beim
Öffnen der Lesepuffer vorgeschrieben wird.
Dies steht lediglich unter Linux zur Verfügung.

Zur Umsetzung sollen die einzelnen Komponenten dem Hauptprogramm als
Programmbibliotheken zur Verfügung gestellt werden.

\subsection{Sprache des Compilers}
\label{sec:language}
Die durch den Compiler zu verarbeitende Sprache ist eine einfache reguläre
Sprache,
deren im folgenden beschrieben wird.

Die Grundmenge der Sprache sind die Ziffern von 0 bis 9,
die Groß- und Kleinbuchstaben des lateinischen Alphabets sowie einige Sonderzeichen.
Daraus ergeben sich folgende Token:
\begin{itemize}
\item \emph{integer}: eine oder mehrere Ziffern
\item \emph{identifier}: mindestens ein Buchstabe, eventuell gefolgt von weiteren Buchstaben oder Ziffern
\item \emph{sign\ldots}: eines der zugelassenen Sonderzeichen, eventuell aus mehreren zusammengesetzt
\item \emph{print}: die Zeichenkette ``print''
\item \emph{read}: die Zeichenkette ``read''
\end{itemize}

Die resultierende Grammatik ist im folgenden tabellarisch dargestellt:

\begin{tabular}{l c l}
\hline \\
letter & ::= & ``A'' \ldots ``B'' \textbar ``a'' \ldots ``b'' \\
digit & ::= & ``0'' \ldots ``9'' \\
sign\ldots & ::= & ``+'' \textbar ``-'' \textbar ``/'' \textbar ``*'' \textbar ``\textless'' \textbar ``\textgreater'' \textbar ``='' \textbar ``\textless!\textgreater'' \textbar ``!'' \textbar ``\&'' \textbar ``;'' \textbar ``('' \textbar ``)'' \textbar ``\{'' \textbar ``\}'' \textbar ``['' \textbar ``]'' \\
integer & ::= & digit digit* \\
identifier & ::= & letter (letter \textbar digit) \\
print & ::= & ``print'' \\
read & ::= & ``read'' \\
\hline
\end{tabular}
