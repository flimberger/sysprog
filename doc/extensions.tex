\chapter{Erweiterungen}

\section{Allgemeine Definitionen}
\label{sec:gendef}

Projektweit wurde der Datentyp \texttt{Datatype} eingeführt,
welcher eine Enumeration aller in der Programmiersprache verfügbaren Datentypen ist.
Dieser Datentyp ist in der projektweit sichtbaren \texttt{defs.h} deklariert,
da er nicht nur vom Compilerprogramm,
sondern auch von der Symboltabellenbibliothek verwendet wird.

Folgende Datentypen stehen in der Sprache zur Verfügung:
\begin{itemize}
\item \texttt{T\_NONE}: kein Datentyp zugewiesen
\item \texttt{T\_ERROR}: Fehler-Typ, es ist ein Typfehler aufgetreten
\item \texttt{T\_INT}: Integer
\item \texttt{T\_INTARR}: Integer-Array
\item \texttt{T\_ARRAY}: Array-Datentyp, eine gültige Array-Indizierung
\end{itemize}

\begin{lstlisting}
typedef enum {
	T_NONE,
	T_ERROR,
	T_INT,
	T_INTARR,
	T_ARRAY
} Datatype;
\end{lstlisting}

\section{Symboltabelle}

Die Grundstruktur der Symboltabelle, das \texttt{Symbol},
wurde um ein Feld für den Datentyp erweitert,
wobei es sich um eine Variable des in Abschnitt~\ref{sec:gendef} beschrieben Typs \texttt{Symbol} handelt.

\begin{lstlisting}
typedef struct Symbol Symbol;
struct Symbol {
	Symbol *next;
	const char *lexem;
	Symboltype symtype;
	Datatype   datatype;
};
\end{lstlisting}
