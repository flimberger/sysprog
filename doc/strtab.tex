\chapter{Stringspeicher}
\label{chap:strtab}

\section{Übersicht}
\label{sec:strtab_overview}

Die Strtab speichert Strings, bzw. Lexeme, die für die Symtab benötigt werden. Sie wird ausschließlich von der Symtab verwendet.

\section{Datenstrukturen und Funktionen}
\label{sec:strtab_datafunc}

Die Strtab stellt die folgenden Datenstrukturen bereit:
\begin{lstlisting}
typedef struct {
	struct strtab_elem *list;
	char *fptr;	/* pointer at free space */
	size_t fsiz;	/* size of free space */
} Strtab;
\end{lstlisting}
Die \texttt{Strtab}-Struktur beschreibt eine einfach verkettete Liste und 
enthält die folgenden Elemente:
\begin{itemize}
\item{list} - Zeigt auf das erste Element der verketten Liste
\item{fptr} - Zeigt auf den nächste freien Speicherplatz
\item{fsiz} - Verfügbarer freier Speicher
\end{itemize}

\begin{lstlisting}
struct strtab_elem {
	struct strtab_elem *next;
	char *data;
};
\end{lstlisting}
Das \texttt{strab\_elem} beschreibt ein Element der Strtab  und hat die folgenden Elemente:
\begin{itemize}
\item{next} - Zeigt auf das folgende Element der Liste
\item{data} - Zeigt auf den eigenen Speicherplatz 
\end{itemize}


Funktionen der Strtab
\begin{itemize}
\item{newelem()}
\item{freeelem()}
\item{addfront()}
\item{addend()}
\item{freeall()}
\item{makestrtab()}
\item{storestr()}
\end{itemize}

Funktion \texttt{newelem()} \\
Die Funktion \texttt{newelem()} erzeugt ein neues \texttt{strtab\_elem} und allokiert die übergebene Größe als Speicher, sodass *data auf diesen Speicher verweist.

Funktion \texttt{freeelem()} \\
\texttt{freeelem()} bekommt ein \texttt{strtab\_elem} übergeben, gibt den allokierten Speicher dieses Elementes frei und das Element selber.

Funktion \texttt{addfront()} \\
Die Funktion \texttt{addfront()} bekommt zwei \texttt{strtab\_elem} übergeben, zum einem das momentane erste Element der Liste sowie das ``neue'' erste Element. Das ``neue'' erste Element verweist nun auf das alte Element und wird zurückgegeben.

Funktion \texttt{addend()} \\
\texttt{addend()} bekommt wie addfront() zwei \texttt{strtab\_elem} übergeben, jedoch soll hier das Element am Ende hinzugefügt werden, also wird abhängig vom ersten übergebenen Element durch die Liste iteriert und am Ende das Element hinzugefüt und zurückgegeben.

Funktion \texttt{freeall()} \\
Die Funktion \texttt{freeall()} iteriert abhängig vom übergebenen \texttt{strtab\_elem} durch die Liste und gibt jedes einzelne Element mit Hilfe von \texttt{freeelem()}

Funktion \texttt{makestrtab()} \\
\texttt{makestrttab()} ist der Konstruktor der \texttt{strtab}, es wird automatisch ein leeres Listenelement erzeugt, auf welches verwiesen wird.

Funktion \texttt{storestr()} \\
Die Funktion \texttt{storestr()} bekommt die Liste sowie den zu speichernden String übergeben, dazu wird ein neues Listen-Element erzeug. Sollte der übergebene String länger als der definierte Speicher sein wird er zum Ende der Liste hinzugefügt sonst am Anfang.