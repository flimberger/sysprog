\chapter{Fehlerbehandlung}
\label{chap:error}

\section{Übersicht}
\label{sec:error_overview}
Die Error Library kümmert sich um die Fehlerbehandlung und enthält die folgenden Funktionen

\begin{itemize}
	\item warn() 	: Fehlerausgabe
	\item die()  	: Abbruch des Programms
	\item panic()	: Core Dump
\end{itemize}

Man findet sie unter src/lib/error.

\section{Funktionen}
\label{sec:error_functions}

Funktion \texttt{vwarn()}

Dies ist die Hauptfunktion der Error Library, sie wird von allen anderen Funktionen aufgerufen.
Übergabeparameter sind der Format-String sowie die Format-Argumente und gibt diese auf dem Standard-Error-Stream aus.

Funktion \texttt{warn()}

Fehlerausgabe Funktion, bekommt den Formatstring sowie Variable Format-Argumente übergeben und gibt diese mit Hilfe der warn()-Funktion aus.

Funktion \texttt{die()}

Programm-Abbruch Funktion, gibt wie vwarn() die Fehler aus und beendet zusätzlich das Programm.

Funktion \texttt{panic()}

``Core-Dump''-Funktion, gibt wie vwarn() die Fehler aus und schreibt zusätzlich ein Speicherabbild auf die Festplatte.


