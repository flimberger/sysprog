\chapter{Symboltabelle}
\label{chap:symtab}

\section{Übersicht}
\label{sec:symtab_overview}
In der Symboltabelle werden die im Quellcode eingelesenen Lexem und deren Typ gespeichert. Zu diesem Zweck wird eine Stringtabelle angelegt, in welcher die benutzten Strings hinterlegt werden.

\section{Datenstrukturen und Funktionen}
\label{sec:symtab_datastructure_function}

Im folgenden werden die Datenstrukturen und Funktionen der Symboltabelle erläutert.

\subsection{Datenstrukturen}
\label{sec:symtab_datastructure}

\begin{lstlisting}
typedef struct Symbol Symbol;
struct Symbol {
	Symbol *next;
	const char *lexem;
	Symboltype type;
};
\end{lstlisting}
Die Datenstruktur \texttt{Symbol} beeinhaltet das Lexem und den Symboltyp. Daneben wird ein Verweis auf das nächste Symbol angegeben, so dass eine einfach verkettete Liste der Symbole entsteht.

\begin{lstlisting}
typedef struct {
	Symbol **symbols;
	Strtab *strtab;
	size_t size;
} Symtab;
\end{lstlisting}
In der Datenstruktur \texttt{Symtab} ist neben einem Verweis auf die Datenstruktur \texttt{Symbol}
noch die Stringtabelle \texttt{Strtab} und die Größe der Symboltabelle enthalten. Diese Datenstruktur stellten den Zugriff auf alle erkannten Symbole und Schlüsselwörter zur Verfügung. 

\subsection{Funktionen}
\label{sec:symtab_functions}

\begin{itemize}
\item makesymtab
\item freesymtab
\item storesym
\item findsym
\end{itemize}
In der ersten Funktion, \texttt{makesymtab} wird als Allokator die Datenstruktur \texttt{Symtab} mit der als Parameter übergebenen Größe erstellt.
Zu diesem Zweck wird der Speicher für die \texttt{Symtab} und die Datenstruktur \texttt{Symbol} alloziert.
Daneben initialisiert \texttt{makesymtab} die Stringtabelle mit Hilfe der Funktion \texttt{makestrtab}. \\
Mit \texttt{freesymtab} wird der Speicher der \texttt{Symtab} und der Datenstruktur \texttt{Symbol} freigegeben.
Zudem wird die Funktion \texttt{freestrtab} aus der \texttt{strtab} aufgerufen, welche den Speicher der Stringtabelle freigibt. \\
Durch \texttt{storesym} und \texttt{findsym} wird die private Funktion \texttt{process} aufgerufen. Der Aufruf dieser privaten Funktion unterscheidet sich hierbei durch den Parameter \texttt{create}.
Dieser Parameter gibt an, ob ein neuer Eintrag angelegt (\texttt{storesym}) oder in der Symboltabelle nach einem bestehenden Eintrag gesucht (\texttt{findsym}) wird.