\chapter{Ein-/Ausgabebibliothek}
\label{chap:buffer}

\section{Übersicht}
\label{sec:buffer_overview}

Die Buffer-Library stellt voll gepufferte Datenein-
und Ausgabe zur Verfügung.
Zur Eingabe wird unter Linux das \texttt{O\_DIRECT}-Flag genutzt,
damit es auf der Seite des Betriessystems nicht zu weiteren Pufferungen kommt.
Dadurch wird Verhindert,
dass es zu Performanceeinbußen aufgrund ungünstiger Mehrfachpufferung kommt.

Das verwendete Puffer-Schema ist Doppelpufferung,
d.h. es gibt zwei Pufferblöcke,
welche im Wechsel zum Lesen oder Schreiben verwendet werden.

Die Puffer ohne zugeordnete Dateien können nicht verwendet werden,
um inkonsistente Dateizustände zu verhinden.
Nicht komplett auf die Platte geschriebene Schreibpuffer werden vor dem Beenden des Prozesses geflushed.

\section{Datenstrukturen und Funktionen}
\label{sec:buf_datafunc}

Folgende Konstanten werden in der Header-Datei definiert:

\begin{lstlisting}
enum {
	Blocksize	= 512,
	Bufsize		= 4 * 1024,
	Bungetsize	= 4,
	NBuf		= 8,		/* # of buffers/process */
	Bfmtstrlen	= 512,		/* Maximum guaranteed len of fmtstr */

	Readbuf		= 1 << 0,	/* status flags positions */
	Writebuf	= 1 << 1,
	Active		= 1 << 2,	/* active if set */

	Dirty		= 0,		/* state of second buffer */
	Clean
};
\end{lstlisting}

\begin{itemize}
\item \texttt{Blocksize} - Blockgröße für die Ausrichtung der Daten
\item \texttt{Bufsize} - Größe eines Pufferblocks
\item \texttt{Bungetsize} - Anzahl der Zeichen, welche mindestens zurückgegangen werden können
\item \texttt{NBuf} - Anzahl der Schreibpuffer pro Prozess
\item \texttt{Bfmtstrlen} - Garantierte Maximallänge für Formatstrings
\item \texttt{Readbuf} - Bitmaske für Schreibpuffer
\item \texttt{Writebuf} - Bitmaske für Lesepuffer
\item \texttt{Active} - Bitmaske für Aktivität/Inaktivität
\item \texttt{Dirty} - Statuskonstante für veränderten Pufferinhalt
\item \texttt{Clean} - Statuskonstante für unveränderten Pufferinhalt
\end{itemize}

Die Puffer-Bibliothek stellt folgende Datenstruktur bereit:
\begin{lstlisting}
typedef struct {
	uchar *nc;	/* pointer to next character */
	uchar *bpb;	/* pointer to beginning of primary buffer */
	uchar *epb;	/* pointer to end of primary buffer */
	uchar *bsb;	/* pointer to beginning of secondary buffer */
	uchar *esb;	/* pointer to end of secondary buffer */
	void *mem;	/* pointer to (aligned) memory */
	size_t size;	/* size of buffers */
	int fd;		/* file descriptor */
	int state;	/* state of second buffer: Clean or Dirty */
	uint flags;	/* flags indicating state of buffer */
} Buffer;
\end{lstlisting}

Die \texttt{Buffer}-Struktur beschreibt einen Puffer und enthält folgende Elemente:

\begin{itemize}
\item \texttt{nc} - Zeiger auf das nächste zu lesende Zeichen im Puffer
\item \texttt{bpb} - Zeiger auf das erste Zeichen des primären Puffers
\item \texttt{epb} - Zeiger auf das letzte Zeichen des primären Puffers
\item \texttt{bsb} - Zeiger auf das erste Zeichen des sekundären Puffers
\item \texttt{esb} - Zeiger auf das letzte Zeichen sekundären Puffers
\item \texttt{mem} - Zeiger auf den Pufferspeicher
\item \texttt{size} - Größe eines Pufferblocks
\item \texttt{fd} - Dem Puffer zugewiesener Dateideskriptor
\item \texttt{state} - Statusflag für den Zustand des Sekundären Puffers
\item \texttt{flags} - Statusflags für den allgemeinen Pufferzustand
\end{itemize}

Der Pufferspeicher muss aufgrund der Verwendung des \texttt{O\_DIRECT}-Flags speziell ausgerichtet sein.
Dafür wird, wie im Abschnitt zu \texttt{makebuf} beschrieben,
die Bibliotheksroutine \texttt{posix\_memalign} verwendet.

Da der Puffer das Doppelpufferschema verwendet,
muss der Status gespeichert werden.
Dies geschieht mittels dem Element \texttt{state}.
Gültige Werte sind die Konstanten \texttt{Dirty} und \texttt{Clean}. \\
Ein Status von \texttt{Clean} bedeutet,
dass der Inhalt des sekundären Puffers gefahrlos überschrieben werden kann. \\
\texttt{Dirty} heißt, dass sich noch nicht gelesene oder geschriebene Daten im Puffer stehen.

Der Allgemeine Pufferzustand wird durch die Variable \texttt{flags} widergespiegelt.
An ihr kann man ablesen,
ob es sich um einen Schreib- oder Lesepuffer handelt.
Gleichzeitige Schreib-/Lesepuffer werden derzeit nicht unterstützt. \\
Eine weiteres Statusbit ist die Aktivität.
Kommt es zu Lese- oder Schreibfehlern,
wird der Pufferstatus auf Inaktiv gesetzt,
um Interaktion mit inkonsistenten Zuständen zu verhindern.

Folgende Routinen werden von der Bibliothek zur Verfügung gestellt:

\begin{itemize}
\item \texttt{makebuf} - Legt eine neue Pufferstruktur mit der spezifizierten Größe an
\item \texttt{initbuf} - Ordnet einen bereits geöffneten Dateideskriptor mit einem Puffer
\item \texttt{bopen} - Öffnet die Datei mit dem gegebenen Modus und gibt einen neu reservierten Puffer zurück
\item \texttt{termbuf} - Trennt die Verbindung zwischen Datei und Puffer
\item \texttt{freebuf} - Gibt den Speicher des Puffers frei
\item \texttt{bclose} - Schließt die Datei des Puffers und gibt anschließend den Speicher frei
\item \texttt{bgetchar} - Gibt das nächste Zeichen aus dem Puffer zurück
\item \texttt{bungetchar} - Geht ein Zeichen im Puffer zurück
\item \texttt{fillbuf} - Liest einen Block aus der Datei in den Puffer ein
\item \texttt{blush} - Schreibt den Inhalt des sekundären Puffers in die Datei
\item \texttt{bbflush} - Schreibt den Inhalt beider Puffer in die Datei
\item \texttt{bputchar} - Schreibt ein einzelnes Zeichen in den Puffer
\item \texttt{bprintf} - Schreibt einen Formatstring in den Puffer
\item \texttt{vbprintf} - Schreibt einen Formatstring und seine vararg-Liste in den Puffer
\end{itemize}

\texttt{makebuf} reserviert eine Buffer-Struktur mittels \texttt{malloc}
und ein Speichersegment	mittels \texttt{posix\_memalign}.
Der Dateideskriptor wird sicherheitshalber auf -1 gesetzt und das flags-Variable auf 0,
damit der leere Puffer nicht verwendet wird. \\
Mit \texttt{initbuf} wird ein bereits bestehender Puffer mit einem bereits geöffneten Dateideskriptor verbunden.
Damit können bereits reservierte Puffer wiederverwendet werden.
Die Routine setzt die Flags entsprechend dem Dateimodus,
setzt die Zeiger entsprechend der Puffergröße
und markiert den Puffer als aktiv und überschreibbar. \\
\texttt{bopen} gibt einen fertigen Puffer für eine Datei zurück.
Der Puffer wird mittels \texttt{makebuf} alloziert,
wobei die Standard-Puffergröße verwendet wird.
Die Datei wird im angegebenen Modus geöffnet
und mit \texttt{initbuf} dem Puffer zugeordnet. \\
Durch einen Aufruf von \texttt{termbuf} wird das Flushen und Deaktivieren des Buffers veranlasst.
Dies geschieht, indem der Dateideskriptor auf -1 und die Flags auf 0 gesetzt werden. \\
Der Speicher eines Puffers wird mittels \texttt{freebuf} freigegeben. \\
\texttt{blcose} flusht den Puffer und schließt die Datei,
danach wird der Speicher freigegeben. \\
\texttt{bgetchar} gibt das nächste aus dem Puffer zurück.
Wenn dieses vor dem Ende des Primärbuffers steht,
wird dieser zurückgegeben und der Zeiger inkrementiert.
Wenn der nächste Character hinter dem Ende des Primärbuffers steht wird geprüft,
ob der Pufferstatus \texttt{Clean} ist.
Ist dies der Fall, wird der Sekundärpuffer neu befüllt.
Schlägt diese Aktion fehl, wird \texttt{EOF} zurückgegeben.
Danach wird überprüft,
ob der Abstand zwischen Beginn und Ende des Sekundärpuffers größer ist als Bungetsize.
Ist dies nicht der Fall,
dann werden Bungetsize Bytes vom Ende des Primärpuffers an dessen Anfang kopiert
und dahinter der Inhalt des Sekundärpuffers.
Das Ende des Primärpuffers wird auf die letzte Position der kopierten Daten gesetzt.
Der Status wird darauf hin wieder auf \texttt{Clean} gesetzt,
da der Sekundärpuffer wieder überschrieben werden kann.
Ansonsten werden die Zeiger auf Beginn und Ende von Primär- und Sekundärpuffer ausgetauscht.
In jedem Fall wird der Zeiger auf das nächste Zeichen auf den Beginn des Primärpuffers gesetzt und ein Sprung an den Beginn	der Routine gemacht,
damit das nächste Zeichen zurückgegeben wird. \\
Mit \texttt{bungetchar} wird das zuletzt gelesene Zeichen zurück in den Puffer ``gepushed''.
Wenn das nächste Zeichen hinter dem Beginn des Primärpuffers steht,
dann wird lediglich der Zeiger darauf dekrementiert.
Ist dies nicht der Fall, aber der Status des Puffers ist nicht \texttt{Clean},
dann werden die Pufferzeiger getauscht und der Zeiger für das nächste Zeichen wird auf das Ende des (neuen) Primärpuffers gesetzt.
In anderen Fällen wird \texttt{EOF} zurückgegeben,
um einen Misserfolg zu signalisieren. \\
Um den sekundären Puffer mit Daten zu füllen,
ruft man \texttt{fillbuf} auf.
Wenn der Status \texttt{Clean} ist,
setzt die Routine den Endezeiger auf das aktuelle Ende der validen Daten.
Wenn ein Lesefehler auftritt,
oder der Status \texttt{Dirty} ist,
wird \texttt{EOF} zurückgegeben,
ansonsten 0.
\texttt{bflush} schreibt die Daten aus dem sekundären Puffer in die entsprechende Datei,
sofern der Pufferstatus \texttt{Dirty ist} und es sich um einen Schreibpuffer handelt.
Wenn die obigen Vorraussetzungen nicht erfüllt werden,
oder der Schreibvorgang erfolgreich war,
wird 0 zurückgegeben.
Wenn ein Fehler beim Schreiben auftritt ist der Rückgabewert \texttt{EOF}. \\
Die zugehörige Funktion \texttt{bbflush} wird hauptsächlich beim Beenden des Prozesses oder dem Schließen des Puffers aufgerufen.
Es veranlasst das Schreiben beider Puffer. \\
Ein einzelnes Zeichen wird mittels \texttt{bputchar} in den Pufer geschrieben.
Wird dabei das letzte Zeichen eines Pufferblocks erreicht,
werden Primär- und Sekundärpuffer vertauscht
und das Statusflag für den Sekundärpuffer auf \texttt{Dirty} gesetzt. \\
Formatstrings werden durch einen Aufruf von \texttt{bprintf} in den Puffer geschrieben.
Wenn angepasste Formatstrings eingeben möchte,
kann dies durch die \texttt{vbprintf} realisiert werden,
welche auch von \texttt{bprintf} verwendet wird.
Dabei können momentan jedoch nur Gesamtlängen von bis zu \texttt{Bfmtstrlen} Zeichen (standardmäßig 512) garantiert werden.
Das Verhalten gleicht den bekannten Bibliotheksfunktionen aus der \texttt{printf}-Familie.
