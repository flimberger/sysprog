\chapter{Scanner}
\label{chap:scanner}

\section{Übersicht}
\label{sec:scanner_overview}
Der Scanner besteht vornehmlich aus einem Automaten,
welcher in der Datei \texttt{acceptor.c} implementiert ist.
Die \texttt{main}-Routine befindet sich zusammen mit einigen Hilfsprozeduren in der Datei \texttt{spc.c},
wo die Randbedingungen für den Scanner vorbereitet werden. \\
Der eigentliche Scan-Vorgang wird das wiederholte aufrufen der Routine \texttt{gettoken} eingeleitet.

\section{Datenstrukturen und Funktionen}
\label{sec:scanner_datafunc}

In der Headerdatei des Scanners wird ein Token als folgende Datenstruktur definiert:

\begin{lstlisting}
typedef struct {
	union {
		long long val;
		Symbol *sym;
		const char *sign;
		char lastchar;
	} data;
	uint	row;
	uint	col;
	Symboltype	type;
} Token;
\end{lstlisting}

Weiterhin werden folgende Routinen deklariert:

\begin{itemize}
\item \texttt{compile} - Leitet den Scanvorgang ein
\item \texttt{gettoken} - Gibt das nächste Token aus der Eingabedatei zurück
\item \texttt{printtoken} - Schreibt eine Menschenlesbare Repräsentation eines Tokens in die Ausgabedatei
\item \texttt{syminit} - Reserviert den Speicherplatz für die Symboltabelle und trägt Schlüsselwörter ein
\item \texttt{symterm} - Gibt die Symboltabelle frei
\end{itemize}

Das Programm wird in der \texttt{main}-Routine gestartet.
Diese setzt den Programmnamen für die Fehlerbehandlungsbibliothek aus \hyperref[chap:error]{Kapitel \ref{chap:error}}
und initialisiert die Symboltabelle mittels \texttt{syminit}.
Danach werden die Ein- und Ausgabedateien entsprechend der Kommandozeilenargumente eingerichtet.
Danach leitet sie durch den Aufruf von \texttt{compile} den Scanvorgang ein.

Die Prozedur \texttt{syminit} initialisiert eine Symboltabelle (siehe \hyperref[chap:symtab]{Kapitel \ref{chap:symtab}})
und registriert ihr Gegenstück \texttt{symterm} durch die Bibliotheksroutine \texttt{atexit},
damit der Speicher beim Beenden des Programms wieder freigegeben wird.

In \texttt{compile} werden die Puffer für die Ein- und Ausgabe geöffnet.
Danach wird eine Schleife betreten,
in welcher alle Tokens aus der Eingabedatei ausgelesen
und in menschenlesbarer Representation in die Ausgabedatei geschrieben werden.
Dazu nutzt sie doe Routine \texttt{printtoken},
welche anhand des Tokentyps eine Ausgabesignatur auswählt
und diese mit den entsprechenden Daten des Tokens in die Ausgabedatei schreibt.

\section{Akzeptor}
\label{sec:scanner_acceptor}

\begin{lstlisting}
enum {
	LEXLEN = 1024 * 4
};

typedef enum {
	LXNEW,
	IDENT,
	INTEG,
	OPADD,	/* + */
	OPSUB,	/* - */
	OPDIV,	/* / */
	OPMUL,	/* * */
	OLESS,	/* < */
	OGRTR,	/* > */
	OASGN,	/* = */
	OUNEQ,	/* <!> */
	OPNOT,	/* ! */
	OPAND,	/* & */
	OTERM,	/* ; */
	OPAOP,	/* ( */
	OPACL,	/* ) */
	OCBOP,	/* { */
	OCBCL,	/* } */
	OBROP,	/* [ */
	OBRCL,	/* ] */
	CMMNT,
	CMEND,
	CUNEQ,
	LXERR,
	MKTOK,
	LXEOF,
	LXEND
} State;
\end{lstlisting}

Der Akzeptor beschreibt einen Automat mit Zuständen und liest mit Hilfe des Buffers die Zeichen ein und baut bei erkanntem Symbol oder Lexem das jeweilige Token.

Die Größe eines Tokens ist abhängig von der oben definierten ``LEXLEN''

\subsection{Funktionen}
\label{sec:acceptor}

Er besitzt folgende Funktionen:
\begin{itemize}
\item{getchar()}
\item{ungetchar()}
\item{mktok()}
\end{itemize}

Sowie jeweils eine eigene Funktion für die einzelnen Zustände in denen die Übergänge realisiert werden.

Funktion \texttt{getchar()} \\
Die Funktion \texttt{getchar()} gibt mit Hilfe des Buffers ein einzelnen Character zurück, welcher dann vom Akzeptor für seine Zustandsübergänge benutzt wird, weiterhin erfolgt hier die Erkennung der jeweiligen Zeile und Spalte.

Funktion \texttt{ungechar()}\\
\texttt{ungetchar()} geht im Buffer einen Character zurück, und kümmert sich wie \texttt{ungetchar()} um die Spalten und Zeilenerkennung.

Funktion \texttt{mktok()} \\
\texttt{mktok()} erzeugt abhängig vom letzten besuchten Zustand ein Token und dessen Daten.



