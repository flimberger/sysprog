\chapter{Typchecker}
\label{chap:typcheck}

\section{Übersicht}
\label{sec:typcheck_overview}
Der Typchecker, welcher in der Datei \texttt{typcheck.c} implementiert ist,
überprüft die korrekte Typisierung des Programmes. 
Aufgerufen wird der Typchecker mit der Funktion \texttt{checkprog}, welche rekursiv die
anderen Funktion aufruft.

\section{Funktionsweise}
\label{sec:typcheck_func}

Der Typchecker wird mit der Funktion \texttt{checkprog} aufgerufen und gestartet, diese Funktion
ruft rekursiv die anderen Funktionen auf.

Für jeden Knotentyp im Parse-Baum gibt es eine Funktion welche die Kinder des Knotents überprüft und dementsprechend deren Funktionen rekursiv aufruft.

In den Funktionen \texttt{checkdecl},\texttt{checkstatement},\texttt{checkindex} und \texttt{checkexp2} erfolgt zusätzlich und abhängig von den jeweiligen Kindsknoten die Überprüfung des Knotentypes sowie die Zuweisung des entpsrechenden Datentyps.