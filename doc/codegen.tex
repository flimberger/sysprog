\chapter{Codegen}
\label{chap:codegen}

\section{Übersicht}
\label{sec:codegen_overview}
Der Codegenerator, welcher in der Datei \texttt{codegen.c} implementiert ist,
erzeugt abhängig von den Knoten des Parse-Baums und deren Datentypen, den definierten Assembler Code. 
Aufgerufen wird der Codegenerator mit der Funktion \texttt{genprog}, welche rekursiv die
anderen Funktion aufruft.

\section{Funktionsweise}
\label{sec:codegen_func}

Nachdem die Typisierung und der Typ-Check erfolgreich abgeschlossen wurde, beginnt der Codegenerator mit der Generierung des definierten Assembler Codes.

Der Codegenerator wird mit der Funktion \texttt{genprog} aufgerufen und gestartet, diese Funktion
ruft rekursiv die anderen Funktionen auf.

Für jeden Knotentyp im Parse-Baum gibt es eine Funktion welche die Kinder des Knotents überprüft und dementsprechend deren Funktionen rekursiv aufruft.

\section{Besonderheiten}
\label{sec:codegen_special}

Bei einem Statement-Knoten wird kein NOP erzeugt, da ein NOP-Befehl keine Funktion besitzt und somit ignoriert werden kann.

Beim Erzeugen eines Labels wird direkt ein NOP erzeugt, da hinter einem Label Programmcode stehen muss, wird hierdurch die Generierung des Codes erleichtert.